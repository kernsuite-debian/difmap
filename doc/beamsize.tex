\documentclass[11pt]{article}
\usepackage{amsmath}
%
% Set the page size.
%
\topmargin -0.5in \headsep 0.251in
\hoffset -0.8in
\marginparwidth 0in \marginparsep 0in
\textheight 9.5in \textwidth 6.5in
\parindent 0pt
\parskip 8pt plus 1pt minus 1pt
\pagestyle{headings}

\newcommand{\bpa}{\theta}
\newcommand{\minor}{\sigma_{\mbox{\scriptsize min}}}
\newcommand{\major}{\sigma_{\mbox{\scriptsize maj}}}
\newcommand{\minorfwhm}{b_{\mbox{\scriptsize min}}}
\newcommand{\majorfwhm}{b_{\mbox{\scriptsize maj}}}
\newcommand{\arctantwo}[2]{\mbox{\,atan2}( #1, #2 )}

\newcommand{\cxx}{c_{xx}}
\newcommand{\cyy}{c_{yy}}
\newcommand{\cxy}{c_{xy}}

\newcommand{\suu}{s_{uu}}
\newcommand{\svv}{s_{vv}}
\newcommand{\suv}{s_{uv}}

\begin{document}

\title{The Determination of the Default Restoring Beam in Difmap}
\author{Martin Shepherd}
\maketitle

\section{Introduction}

Observations of the sky with an interferometer sample the Fourier
transform of the sky at discrete points. This Fourier transform plane
is known as the UV plane, named after the U and V coordinates that are
used to label its axes. After accumulating many samples at different
locations in the UV plane, an image may be formed by gridding these
samples onto a regular two-dimensional array and taking the inverse
Fourier transform. The result is an image of the sky that has been
convolved with the PSF of the interferometer. The PSF, known as the
dirty beam, is the inverse Fourier transform of the sampling weights
in the UV plane. The dirty beam, tends to be very irregular due to
incomplete sampling of the UV plane, so the initial image of the sky
can be hard to interpret, with each bright feature in the image being
accompanied by side-lobes.

To obtain an image that is easier to analyze than the dirty image, a
number of deconvolution schemes have been developed. These aim of
re-convolving the image with a much simpler beam that doesn’t have
side-lobes, such as an elliptical Gaussian, but with similar average
resolution to the irregular synthesized beam.

The simplest deconvolution scheme is known as CLEAN. This decomposes
the dirty image into a set of delta functions, each one representing a
separate bright point-like feature in the image, plus a residual image
of the background noise. The delta functions, known as CLEAN
components, are then convolved with the chosen elliptical Gaussian
beam, known as the restoring beam, and the resulting images are added
back to the residual image of the noise. The final result is known as
the restored or clean image.

This document describes one way to choose the dimensions and
orientation of an elliptical Gaussian restoring beam that best matches
the resolution of the dirty beam. An obvious way to do this would be
to perform a least-squares fit of an elliptical Gaussian to an image
of the dirty beam. However this is not as straight forward as it might
seem, because in general the dirty beam does not look much like a
Gaussian, except very close to the center.

The method used in Difmap, which was inherited from the earlier
Caltech VLBI programs, takes note of the fact that the center of the
dirty beam is the only part of the synthesized beam that tends to look
somewhat Gaussian. What it does is calculate the elliptical Gaussian
that has the same curvature as the dirty beam at the center of the
image. Image plane derivatives are trivial to calculate in the Fourier
plane, so in practice this is done while calculating the weights used
to grid the observed data in the UV plane.

\section{Characterizing the clean beam}

Let $f(\alpha,\beta)$ be an elliptical Gaussian beam with a peak value
of unity, where $\alpha$ is a coordinate along the major axis, whose
length is the Gaussian standard deviation $\major$, and $\beta$ is the
coordinate along the minor axis, whose length is the Gaussian standard
deviation, $\minor$.

In the $(\alpha,\beta)$ coordinate system, the two-dimensional
elliptical gaussian PSF can be written:

\begin{equation}
\label{gaussian_eqn}
  f(\alpha,\beta) = \exp(-0.5 (\alpha^2 / \major^2 + \beta^2 / \minor^2))
\end{equation}

On the sky the major axis of the beam is rotated anticlockwise east of
north by an angle $\bpa$. Within Difmap, however, images of the sky
are stored with coordinates of $(x,y)$, where $x$ goes from west to
east. In this coordinate system the beam is rotated clockwise by the
angle $\bpa$.

Given a sky coordinate, $(x,y$), the corresponding position along the
major and minor axes of the elliptical Gaussian are given by rotating
the $x$ and $y$ clockwise by the angle $\bpa$, as follows.

\begin{equation}
\begin{pmatrix}
\beta \\
\alpha
\end{pmatrix}
=
\begin{pmatrix}
    \cos{\bpa} & -\sin{\bpa}\\
    \sin{\bpa} &  \cos{\bpa}
\end{pmatrix}
\begin{pmatrix}
x \\
y
\end{pmatrix}
\end{equation}

\begin{eqnarray}
  \beta &=&  x\cos{\bpa} - y\sin{\bpa} \\
  \alpha &=&  x\sin{\bpa} + y\cos{\bpa}
\end{eqnarray}

If we substitute these equations for $\alpha$ and $\beta$ into
equation~\ref{gaussian_eqn}, we obtain the following equation for the
beam in the $(x,y)$ coordinate system.

\begin{equation}
\label{beam_xy_eqn}
    s(x,y) = \exp(-0.5 (x\sin{\bpa} + y\cos{\bpa})^2 / \major^2
                  -0.5 (x\cos{\bpa} - y\sin{\bpa})^2 / \minor^2)
\end{equation}

\section{The curvature of the Gaussian at the origin}

The second derivative of $s(x,y)$ with respect to $x$ is is as follows.

\begin{multline}
    \frac{d^2s}{d x^2} = ( (-y\cos{\bpa}\sin{\bpa} - x\sin{\bpa}^2) / \major^2 +
                           ( y\cos{\bpa}\sin{\bpa} - x\cos{\bpa}^2) / \minor^2
              )^2  s(x,y)\\ + (-\sin{\bpa}^2/\major^2 - \cos{\bpa}^2/\minor^2)  s(x,y)
\end{multline}

Similarly, the second derivative of $s(x,y)$ with respect to $y$ is as follows.

\begin{multline}
    \frac{d^2s}{d y^2} = ( (y\cos{\bpa}^2 - x\cos{\bpa}\sin{\bpa}) / \major^2 +
                (y\sin{\bpa}^2 + x\cos{\bpa}\sin{\bpa}) / \minor^2
              )^2  s(x,y)\\ + (-\cos{\bpa}^2/\major^2 - \sin{\bpa}^2/\minor^2)  s(x,y)
\end{multline}

Finally, the second derivative of $s(x,y)$ with respect to $dx dy$ is
as follows.

\begin{multline}
      \frac{ds}{dx dy} = ((-y\cos{\bpa}\sin{\bpa} - x\sin{\bpa}^2) / \major^2 +
                 ( y\cos{\bpa}\sin{\bpa} - x\cos{\bpa}^2) / \minor^2
                )\times \\ ((-x\cos{\bpa}\sin{\bpa} - y\cos{\bpa}^2) / \major^2 +
                     ( x\cos{\bpa}\sin{\bpa} - y\sin{\bpa}^2) / \minor^2
                    ) s(x,y)\\ + (-\cos{\bpa}\sin{\bpa} / \major^2 + \cos{\bpa}\sin{\bpa}/\minor^2) s(x,y)
\end{multline}

If we evaluate each of the above derivatives at the origin,
$(x=0,y=0)$, then we obtain the following for the derivatives of the
Gaussian at the origin of the image plane.

\begin{align}
\label{cxx_eqn}
    \cxx &= d^2s(0,0)/dx^2 &= -\sin{\bpa}^2 / \major^2 - \cos{\bpa}^2 / \minor^2\\
\label{cyy_eqn}
    \cyy &= d^2s(0,0)/dy^2 &= -\cos{\bpa}^2 / \major^2 - \sin{\bpa}^2 / \minor^2\\
\label{cxy_eqn}
    \cxy &= d^2s(0,0)/dxdy &= 0.5  \sin{2\bpa}  (1/\minor^2 - 1/\major^2)
\end{align}

Note that the two curvatures (equations~\ref{cxx_eqn} and
\ref{cyy_eqn}) are negative, as expected for the curvature of a peak.

\section{Obtaining Gaussian parameters from the second derivatives}

As a first step towards determining the Gaussian parameters, $\major$,
$\minor$ and $\bpa$, we combine equations~\ref{cxx_eqn},\ref{cyy_eqn}
and \ref{cxy_eqn} as follows.

\begin{align}
    \cxx + \cyy &= -1/\major^2 - 1/\minor^2 \\
    \cxx - \cyy &= -\cos{2\bpa}  (1/\minor^2 - 1/\major^2)\\
    (2 \cxy)^2 + (\cxx-\cyy)^2 &= (1/\minor^2 - 1/\major^2)^2
\end{align}

In the above equations, note that we are aiming to obtain $(1/\minor^2
- 1/\major^2)$, which is positive, whereas $(1/\major^2 - 1/\minor^2)$
should be negative. This becomes important below where we would
otherwise lose a negative by square-rooting the square of this term.

Solving the above equations for the Gaussian parameters, we obtain the
following.

\begin{eqnarray}
  \label{bpa_im_eqn}
    \bpa &=& 0.5  \arctantwo{2\cxy}{\cyy-\cxx}\\
  \label{major_im_eqn}
    \major &=& \sqrt{\frac{2}{-\left(\sqrt{4\cxy^2 + (\cxx-\cyy)^2} + \cxx+\cyy\right)}}\\
  \label{minor_im_eqn}
    \minor &=& \sqrt{\frac{2}{\sqrt{4\cxy^2 + (\cxx-\cyy)^2} - (\cxx+\cyy)}}
\end{eqnarray}

\section{Obtaining derivatives via the Fourier plane}

We can obtain $\cxx$, $\cyy$ and $\cxy$ at the origin by performing a simple
operation to the Fourier transform of the beam.

If the Fourier transform of $s(x,y)$ is denoted $S(u,v)$, then the
image plane derivatives needed in the previous section can be obtained
as follows.

\begin{align}
  d^2s/dx^2 &<=> -4\pi^2  u^2  S(u,v)\\
  d^2s/dy^2 &<=> -4\pi^2  v^2  S(u,v)\\
  ds/dxdy &<=> -4\pi^2 u v  S(u,v)
\end{align}

The value at the origin of an inverse Fourier transform is the
integral or sum of the values in the Fourier transform, so we can
calculate the desired derivatives in the image plane as follows.

\begin{align}
  \cxx &= -4 \pi^2  \sum{u^2  S(u,v)} \\
  \cyy &= -4 \pi^2  \sum{v^2  S(u,v)} \\
  \cxy &= -4 \pi^2  \sum{u v  S(u,v)}
\end{align}

To simplify the subsequent equations, we define the following values.

\begin{align}
  \suu &= \sum{u^2  S(u,v)} \\
  \svv &= \sum{v^2  S(u,v)} \\
  \suv &= \sum{u v  S(u,v)}
\end{align}

In terms of these values, the derivatives can now be written as
follows.

\begin{align}
\label{cxx_uv_eqn}
  \cxx &= -4\pi^2  \suu \\
\label{cyy_uv_eqn}
  \cyy &= -4\pi^2  \svv \\
\label{cxy_uv_eqn}
  \cxy &= -4\pi^2  \suv
\end{align}

\section{Computing the Gaussian parameters via the Fourier plane}

If we insert the derivatives of equations~\ref{cxx_uv_eqn},
\ref{cyy_uv_eqn} and \ref{cxy_uv_eqn} into equations~\ref{bpa_im_eqn},
\ref{minor_im_eqn} and \ref{major_im_eqn}, we obtain the following
equations for the Gaussian parameters.

\begin{align}
  \bpa &= -0.5  \arctantwo{2 \suv}{\suu-\svv}\\
  \minor &= 1 / \sqrt{2} / \pi / \sqrt{\sqrt{4 \suv^2 + (\suu-\svv)^2} + (\suu+\svv)}\\
  \major &=  1 / \sqrt{2} / \pi / \sqrt{-(\sqrt{4 \suv^2 + (\suu-\svv)^2} - (\suu+\svv))}
\end{align}

In difmap, the dimensions of the Gaussian beam are specified as Full
Widths at Half Maximum (FWHM). We can obtain those by scaling the
standard deviations given above by $\sqrt{8 \ln{2}}$.

\begin{align}
  \minorfwhm &= \sqrt{4 \ln{2}}/\pi / \sqrt{\suu+\svv + \sqrt{4 \suv^2
      + (\suu-\svv)^2}} \\
  \majorfwhm &= \sqrt{4 \ln{2}}/\pi / \sqrt{\suu+\svv - \sqrt{4
         \suv^2 + (\suu-\svv)^2}}
\end{align}

\section{The algorithm historically used in Difmap}

In Difmap versions up to 2.5b (25 Nov 2018)), the algorithm difmap
used was inherited verbatim from the VLBI programs, which used the
following equations:

\begin{align}
  k &= 0.7 \\
  \bpa &= -0.5  \arctantwo{2 \suv}{\suu - \svv} \\
  \minorfwhm &= k/\sqrt{2} / \sqrt{\suu+\svv + \sqrt{4 \suv^2 + (\suu-\svv)^2}} \\
  \majorfwhm &= k/\sqrt{2} / \sqrt{\suu+\svv - \sqrt{4 \suv^2 + (\suu-\svv)^2}}
\end{align}

From the correct equations of the previous section, $k$ ought to have
had the following value.

\begin{align}
  k_{\mbox{\scriptsize true}} &= \sqrt{8 \ln{2}}/\pi \\
               &= 0.7495625
\end{align}

This is 6.6\% larger than the value used in Difmap, such that the
default Gaussian beam used to restore images in Difmap was been 6.6\%
too small.

This was corrected in Difmap version 2.5c (20 May 2019).

\end{document}
